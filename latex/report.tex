\documentclass[conference]{IEEEtran}
\usepackage[UTF8]{ctex}
\usepackage{graphicx}
\usepackage{listings} % 引入宏包
\usepackage{float}
\usepackage{subcaption}
\usepackage{xcolor}   % 引入颜色宏包,用于代码块着色
\usepackage{amsmath}
\usepackage{hyperref}
\usepackage{enumitem}
\usepackage{algorithm}
\usepackage{algpseudocode}
\renewcommand{\algorithmicrequire}{\textbf{输入:}}
\renewcommand{\algorithmicensure}{\textbf{输出:}}

% ISEE -> IEEE
\title{基于NumPy手写深度强化学习框架的迷宫AI设计}

\author{\IEEEauthorblockN{黄鹤翔    高恒斌  都奕宁} 
\IEEEauthorblockA{信息与电子工程学院\\
浙江大学, 杭州, 中国}
}

\begin{document}
\maketitle
% 显示页码(默认居中页脚),适用于需要打印或评审时查看页码
\pagestyle{plain}
\begin{abstract}
本报告详细阐述了一个基于深度强化学习(Deep Q-Network, DQN)的迷宫自动寻路及对战AI系统的设计与实现。项目的核心挑战在于不依赖现有的深度学习框架(如PyTorch或TensorFlow),而是仅使用NumPy库从底层构建卷积神经网络(CNN)及优化器。报告首先介绍了迷宫生成算法(DFS与Prim),随后深入探讨了自定义深度学习框架\texttt{simple\_nn}的实现细节,包括\texttt{im2col}卷积加速、反向传播算法及Adam优化器。在此基础上,搭建了DQN模型,设计了包含墙壁信息、位置信息及访问记录的7通道状态空间,并通过经验回放(Experience Replay)和目标网络(Target Network)机制稳定训练过程。最后,展示了AI在迷宫寻路测试及人机对战模式中的表现,验证了手写框架及算法的有效性。
\end{abstract}

\begin{IEEEkeywords}
深度强化学习, DQN, 卷积神经网络, NumPy, 迷宫生成, 自动寻路
\end{IEEEkeywords}

\lstset{                    % 设置代码显示风格
    language=Python,        % 代码语言修正为 Python
    basicstyle=\ttfamily\small,   % 基本字体设置
    keywordstyle=\color{blue},  % 关键字颜色
    commentstyle=\color{green!60!black}, % 注释颜色
    stringstyle=\color{red},    % 字符串颜色
    showstringspaces=false, % 不显示空格符
    breaklines=true,
    frame=single,
    numbers=left,
    numberstyle=\tiny\color{gray},
    extendedchars=false,    % 关键设置:解决中文在 listings 中的报错问题
    mathescape=false,       % 确保特殊符号不被当作数学公式
}

\section{引言}
项目的设计灵感源于B站up\@ 河南葫季果和\@ L\_Shy\_P的视频,视频中出现了一个近乎无敌的ai,能够追踪敌人躲避子弹。

本项目旨在旨在构建一个具备自主学习能力的坦克对战AI系统。与常规项目不同,本项目的核心要求是“去框架化”,即不调用成熟的深度学习库,而是利用Python和NumPy从零实现神经网络的底层算子(卷积、全连接、激活函数)及训练机制。这不仅考验对强化学习算法的理解,更要求对神经网络反向传播及矩阵运算有深入的掌握。

\section{设计任务和要求}
本项目的具体设计任务和要求如下:
\begin{enumerate}
    \item \textbf{迷宫环境构建}:实现基于深度优先搜索(DFS)和Prim算法的随机迷宫生成器,支持可视化展示。
    \item \textbf{底层框架实现}:仅使用NumPy库,实现卷积层(Conv2d)、全连接层(Linear)、激活函数(ReLU)、损失函数(MSELoss)及优化器(Adam)。
    \item \textbf{AI算法实现}:基于手写框架,在简单的环境中搭建DQN框架训练ai,在复杂的环境中使用遗传算法迭代ai。
    \item \textbf{人机对战模式}:集成训练好的模型,实现玩家与AI在迷宫中的实时对战(坦克大战模式)。
\end{enumerate}

\section{算法原理}

\subsection{迷宫生成算法-DFS}
DFS 迷宫生成算法本质上是一种递归回溯过程。它从起始格子出发,随机选择一个未访问的相邻格子进行“打通”(移除中间墙壁),然后递归地对该新格子执行相同操作。当当前格子的所有邻居均已被访问时,算法回退至上一层,直至遍历完整个网格。
\begin{itemize}
\item 将每个网格单元视为图中的一个节点;
\item 初始时所有节点标记为“未访问”,所有相邻节点间存在“墙”;
\item 从任意起点(如左上角 (0,0))开始 DFS 遍历;
\item 每次移动到新节点时,移除当前节点与目标节点之间的墙,并将目标节点加入连通图;
\item 递归完成后,整个网格形成一棵生成树,对应一个无环连通迷宫。
\end{itemize}

\subsection{迷宫生成算法-prim}
Prim 算法原本用于求解最小生成树(Minimum Spanning Tree, MST)。在迷宫生成中,我们将其改造为随机 Prim 算法:不考虑边权,而是随机选择待扩展的边。\par

算法维护一个“边界集合”(frontier set),包含所有与已生成区域相邻但尚未加入的格子。每一步从边界集中随机选取一个格子,将其与已生成区域中的某个邻居连接(打通墙壁),并将该格子的新邻居加入边界集。重复此过程直至覆盖全部格子。\par

Prim 生成的迷宫通常具有更多短分支和较短的主路径,整体结构更“开阔”,有利于测试智能体在多岔路口的决策能力。
\begin{itemize}
\item 将网格视为完全图,每条潜在通道(相邻格子间)是一条边;
\item 任选一个起始格子加入生成树;
\item 将其所有相邻格子加入“边界集”;
\item 循环:从边界集中随机选一格子,随机选择其一个已在生成树中的邻居,打通两者之间的墙,并将该格子纳入生成树,同时将其未访问邻居加入边界集;
\item 直至所有格子被纳入。
\end{itemize}

DFS生成的迷宫通常具有长而曲折的走廊和较少的分支,Prim 生成的迷宫通常具有更多短分支和较短的主路径
\begin{figure}[H]
    \centering
    \includegraphics[scale = 0.2]{figure/dfs.png}
    \caption{DFS生成迷宫}
\end{figure}
\begin{figure}[H]
    \centering
    \includegraphics[scale = 0.2]{figure/prim.png}
    \caption{prim生成迷宫}
\end{figure}

\subsection{解迷宫算法-A*}
A*(A-Star)算法是一种广泛应用于路径规划和图遍历的启发式搜索算法。它结合了 Dijkstra 算法的完备性与贪心最佳优先搜索的效率,能够高效地找到从起点到终点的最短路径。\par

A* 算法通过评估函数 f(n) 来决定下一个要探索的节点:
$$
f(n)=g(n)+h(n)
$$
其中:
\begin{itemize}
\item g(n):从起点到当前节点 n 的实际代价。
\item h(n):从当前节点 n 到目标点的启发式估计代价。
\end{itemize}
在本项目中使用曼哈顿距离$abs(x_1 - x_2) + abs(y_1 - y_2)$进行启发式搜索。
我们在初始化时将起点(0,0)加入open列表中,然后进入循环,直到open列表为空或找到终点为止。在每一步迭代中,我们从open列表中选择f值最小的节点作为当前节点。然后检查当前节点是否为目标节点,如果是则结束循环;如果不是目标节点,则将其标记为已处理并将其相邻的未处理节点加入open列表中。同时,将已处理的节点加入closed列表中以避免重复处理。
\begin{figure}[H]
    \centering
    \includegraphics[scale = 0.2]{figure/astar.png}
    \caption{A*算法解迷宫}
\end{figure}

\subsection{自定义深度学习框架 (simple\_nn)}
为了替代PyTorch,我们实现了\texttt{simple\_nn.py}。
\subsubsection{卷积层实现}
卷积操作的核心在于高效地提取局部特征。为了避免低效的多重循环,采用了\texttt{im2col}(image to column)技术,将输入特征图展开为矩阵,从而将卷积运算转化为矩阵乘法(GEMM)。
\begin{equation}
    Y = W_{col} \times X_{col} + b
\end{equation}
其中 $X_{col}$ 是展开后的输入矩阵,$W_{col}$ 是重排后的权重矩阵。

\subsubsection{优化器}
实现了Adam优化器,结合了动量法(Momentum)和RMSProp的优点,自适应调整学习率。
\begin{align}
    m_t &= \beta_1 m_{t-1} + (1-\beta_1) g_t \\
    v_t &= \beta_2 v_{t-1} + (1-\beta_2) g_t^2 \\
    \theta_{t+1} &= \theta_t - \frac{\eta}{\sqrt{\hat{v}_t} + \epsilon} \hat{m}_t
\end{align}

\subsection{AI训练-DQN}
DQN (Deep Q-Network) 将卷积神经网络与Q-Learning结合,解决了高维状态空间的强化学习问题。
\subsubsection{核心机制}
\begin{itemize}
    \item 经验回放 (Experience Replay):构建一个回放池 $D$,存储转移样本 $(s, a, r, s', done)$。训练时随机采样一个小批量(Batch),打破了数据间的相关性,提高了训练的稳定性。
    \item 目标网络 (Target Network):引入一个参数滞后的目标网络 $\hat{Q}$ 计算目标值,避免了“自举”(Bootstrapping)导致的目标值震荡。目标网络每一定时间从策略网络中复制参数。
\end{itemize}
通过贝尔曼最优方程来计算期望Q值,即执行该动作后获得的即时奖励 r 加上对未来最大 Q 值的折扣期望
$$
Q^*(s, a) = \mathbf{E}*{s' \sim \mathcal{P}} \left[ r + \gamma \max*{a'} Q^*(s', a') \right]
$$
目标函数为最小化均方误差:
\begin{equation}
    L(\theta) = \mathbf{E}\left[ \left( r + \gamma \max_{a'} \hat{Q}(s', a'; \theta^-) - Q(s, a; \theta) \right)^2 \right]
\end{equation}

\begin{figure}[H]
    \centering
    \includegraphics[scale = 0.2]{figure/dqn.png}
    \caption{dqn算法流程结构}
\end{figure}

\subsubsection{Q值估计网络设计-输入层}
输入层设计为 $7 \times H \times W$ 的张量,包含7个通道:
\begin{itemize}
    \item 通道 0-3 (墙壁信息):分别编码每个格子上、下、左、右四个方向的墙壁存在情况
    \item 通道 4-5 (位置信息):分别使用One-hot编码智能体位置和目标位置
    \item 通道 6 (历史轨迹):记录智能体访问过的路径,赋予智能体记忆
\end{itemize}

\subsubsection{Q值估计网络设计-卷积层}
网络包含4层卷积层,通道数分别为 32, 64, 128, 128,主要为卷积层与relu激活层的循环。
网络中没有加入传统CNN网络中常见的池化层,主要是因为池化层的加入会模糊智能体与目标点的位置信息,
影响训练效果。\par
通过堆叠4层卷积层,深层神经元的感受野逐渐扩大,能够感知更大范围的迷宫结构,从而规划长距离路径。
通道数逐层增加(32 $\to$ 128),使得网络能够组合低级几何特征.
\begin{table}[htbp]
\centering
\caption{卷积神经网络各层参数配置}
\begin{tabular}{|l|c|c|c|c|c|}
\hline
\textbf{} & \textbf{输入通道数} & \textbf{输出通道数} & \textbf{卷积核大小} & \textbf{步长} & \textbf{填充} \\ \hline
第一层    & 7                   & 32                  & 3                    & 1           & 1           \\ \hline
第二层    & 32                  & 64                  & 3                    & 1           & 1           \\ \hline
第三层    & 64                  & 128                 & 3                    & 1           & 1           \\ \hline
第四层    & 128                 & 128                 & 3                    & 1           & 1           \\ \hline
\end{tabular}
\end{table}

\subsubsection{Q值估计网络设计-全连接层}
卷积层输出展平后,连接3层全连接层(1024 $\to$ 512 $\to$ 4),提供了强大的非线性拟合能力,将提取的迷宫特征转化为上、下、左、右四个离散动作的Q值。

\subsection{超参数设计}
\begin{lstlisting}
# Hyperparameters
WIDTH, HEIGHT = 10, 10
EPISODES = 10000 
BATCH_SIZE = 32
GAMMA = 0.9 # 对未来奖励的关心程度,越大越关心
# 使用线性下降的EPSILON,EPSILON越大越倾向于随机探索
EPSILON_START = 1.0
EPSILON_END = 0.05
EPSILON_DECAY_EPISODES = 4000
LR = 0.0001 
TARGET_UPDATE = 200 #目标网络的更新频率
MEMORY_SIZE = 50000 #经验池的大小
\end{lstlisting}

\subsection{AI训练-遗传算法}
遗传算法是一种受达尔文生物进化论启发的元启发式优化算法。它模拟了自然界“物竞天择,适者生存”的过程,通过在解空间中维护一个种群(Population),并利用选择(Selection)、交叉(Crossover)和变异(Mutation)等遗传算子,迭代地演化出越来越优的解决方案。\par

在这个项目中,我们使用一个全连接神经网络来对智能体的下一步动作进行决策,通过遗传算法迭代更新神经网络的参数。
同时在训练过程中,我们借鉴了DQN的训练思路,每隔一段时间将训练目标的参数从智能体中复制下来,以求获得更加强大的智能体。

\begin{itemize}
    \item 选择算子:锦标赛选择
    \item 交叉算子:均匀交叉
    \item 变异算子:高斯扰动
    \item 进化策略:精英保留,将每一代中适应度最高的前10\%个体(精英)直接复制到下一代,不经过交叉和变异
\end{itemize}

\section{主要仪器设备}
\begin{itemize}
    \item \textbf{开发语言}:Python 3.10
    \item \textbf{核心库}:NumPy (用于矩阵运算), Matplotlib (用于可视化)
    \item \textbf{硬件环境}:AMD CPU
    \item \textbf{操作系统}:Linux
\end{itemize}

\section{设计结果记录与分析}

\section{结论}


\section{参考文献}
\begin{enumerate}
    \item https://www.bilibili.com/video/BV17EswzLEKw/
\end{enumerate}

\end{document}
