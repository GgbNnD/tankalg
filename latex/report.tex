\documentclass[conference]{IEEEtran}
\usepackage[UTF8]{ctex}
\usepackage{graphicx}
\usepackage{listings} % 引入宏包
\usepackage{float}
\usepackage{subcaption}
\usepackage{xcolor}   % 引入颜色宏包,用于代码块着色
\usepackage{amsmath}
\usepackage{hyperref}
\usepackage{enumitem}
\usepackage{algorithm}
\usepackage{algpseudocode}
\renewcommand{\algorithmicrequire}{\textbf{输入:}}
\renewcommand{\algorithmicensure}{\textbf{输出:}}

% ISEE -> IEEE
\title{离散与连续环境下的智能体决策系统:基于DQN与行为树的坦克AI研究}

\author{\IEEEauthorblockN{黄鹤翔\quad 高恒斌\quad 都奕宁} 
\IEEEauthorblockA{信息与电子工程学院\\
浙江大学, 杭州, 中国}
}

\begin{document}
\maketitle
% 显示页码(默认居中页脚),适用于需要打印或评审时查看页码
\pagestyle{plain}
\begin{abstract}
本报告详细阐述了一个基于深度强化学习(DQN)与启发式决策的迷宫自动寻路及对战AI系统的设计与实现。
项目的核心挑战在于不依赖现有的深度学习框架,而是仅使用NumPy库从底层构建卷积神经网络
及优化器。报告首先介绍了迷宫生成算法(DFS、Prim)及经典启发式寻路算法(A*),随后深入探讨了
自定义深度学习框架\texttt{simple\_nn}的实现细节,包括\texttt{im2col}卷积
加速、反向传播算法及Adam优化器。在此基础上,搭建了DQN模型,设计了包含墙
壁信息、位置信息及访问记录的7通道状态空间,并通过经验回放和目标网络
机制稳定训练过程。同时,探索了基于遗传算法的神经网络进化策略,通过锦标赛选择与精英保留机制优化智能体决策。此外,针对连续物理环境下的复杂对战需求,设计了基于分层行为树
的智能决策系统。该系统引入了基于Liang-Barsky算法的通视性与危险势场分析、结合“拉线法”
的混合路径规划及动态规避策略,有效解决了连续空间下的碰撞检测与精准操控难题。
最后,展示了AI在迷宫寻路测试及人机对战模式中的表现,验证了手写框架及决策树AI算法的有效性。
\end{abstract}

\begin{IEEEkeywords}
深度强化学习, DQN, 卷积神经网络, NumPy, 迷宫生成, A*算法, 自动寻路, 启发式决策, 行为树, 混合路径规划, 危险势场
\end{IEEEkeywords}

\lstset{                    % 设置代码显示风格
    language=Python,        % 代码语言修正为 Python
    basicstyle=\ttfamily\small,   % 基本字体设置
    keywordstyle=\color{blue},  % 关键字颜色
    commentstyle=\color{green!60!black}, % 注释颜色
    stringstyle=\color{red},    % 字符串颜色
    showstringspaces=false, % 不显示空格符
    breaklines=true,
    frame=single,
    numbers=left,
    numberstyle=\tiny\color{gray},
    extendedchars=false,    % 关键设置:解决中文在 listings 中的报错问题
    mathescape=false,       % 确保特殊符号不被当作数学公式
}

\section{引言}
项目的设计灵感源于B站up\@ 河南葫季果和\@ L\_Shy\_P的视频,视频中出现了一个近
乎无敌的AI,能够追踪敌人躲避子弹。

本项目旨在构建一个具备自主学习能力的坦克对战AI系统。与常规项目不
同,本项目的核心要求是“去框架化”,即不调用成熟的深度学习库,而是
利用Python和NumPy从零实现神经网络的底层算子及训练机制。同时,本项目还尝试了遗传算法(Genetic Algorithm)作为另一种优化手段,通过模拟生物进化过程来迭代神经网络参数。这不仅考验对
强化学习算法的理解,更要求对神经网络反向传播、矩阵运算及优化算法有深入的掌握。

此外,为了应对更加复杂的连续物理环境对战需求,本项目还设计了一套基于
启发式规则与行为树的智能决策系统。传统的DQN模型虽然在离散网格中表现优异,
但在处理连续坐标、任意角度旋转及精细碰撞检测时面临状态空间爆炸的问题。
因此,本项目引入了“Smart AI”架构,不再依赖强化学习,而是结合了
危险势场分析、混合路径规划及动态规避算法。
这种启发式决策的设计,旨在全面探索不同技术路线在
游戏AI中的应用潜力,既实现了离散环境下的端到端学习,又保证了连续环境下的高水平竞技表现。

\section{设计任务和要求}
本项目的具体设计任务和要求如下,涵盖了从底层框架到上层决策的全栈开发:
\begin{enumerate}
    \item \textbf{迷宫环境构建}:
    \begin{itemize}
        \item 实现基于深度优先搜索和随机Prim算法的迷宫生成器,支持生成不同拓扑结构的地图。
        \item 实现A*启发式寻路算法,作为AI路径规划的基准。
    \end{itemize}
    
    \item \textbf{底层深度学习框架实现}:
    \begin{itemize}
        \item 仅使用NumPy库,从零构建神经网络核心算子,包括全连接层、卷积层及激活函数。
        \item 实现\texttt{im2col}技术以加速卷积运算,并推导实现各层的反向传播梯度计算。
        \item 实现Adam自适应优化器及MSELoss损失函数,支持网络的参数更新。
    \end{itemize}

    \item \textbf{离散环境强化学习AI}:
    \begin{itemize}
        \item 构建基于CNN的Q值估计网络,设计包含墙壁、位置及轨迹信息的7通道状态空间。
        \item 实现经验回放与目标网络机制,解决训练不稳定的问题。
        \item 在离散网格环境中训练智能体,实现从随机探索到最优路径规划的自主学习。
    \end{itemize}

    \item \textbf{遗传算法进化策略}:
    \begin{itemize}
        \item 构建全连接神经网络作为决策模型,探索非梯度下降类的优化方法。
        \item 实现基于锦标赛选择、均匀交叉和高斯变异的遗传算法,用于神经网络参数的进化迭代。
    \end{itemize}

    \item \textbf{连续环境物理引擎适配}:
    \begin{itemize}
        \item 升级游戏引擎以支持浮点坐标与任意角度旋转。
        \item 实现基于分离轴定理或Liang-Barsky算法的射线-OBB碰撞检测,用于子弹判定与视线分析。
    \end{itemize}

    \item \textbf{连续环境启发式智能AI}:
    \begin{itemize}
        \item \textbf{危险感知}:设计危险评分模型,基于物理预测计算未来时间窗口内的碰撞风险。
        \item \textbf{决策系统}:构建分层行为树,实现生存优先、动态规避、战术进攻与路径巡逻的状态切换。
        \item \textbf{运动控制}:设计“网格BFS + String Pulling拉线法”的混合路径规划算法,并实现基于PID思想的平滑路径跟踪控制。
    \end{itemize}
    
    \item \textbf{人机对战系统集成}:集成上述模型,实现流畅的实时对战功能,并提供AI决策的可视化调试。
\end{enumerate}

\section{算法原理}

\subsection{迷宫生成算法-DFS}
DFS 迷宫生成算法本质上是一种递归回溯过程。它从起始格子出发,
随机选择一个未访问的相邻格子进行“打通”(移除中间墙壁),
然后递归地对该新格子执行相同操作。当当前格子的所有邻居
均已被访问时,算法回退至上一层,直至遍历完整个网格。
\begin{itemize}
\item 将每个网格单元视为图中的一个节点;
\item 初始时所有节点标记为“未访问”,所有相邻节点间存在“墙”;
\item 从任意起点(如左上角 (0,0))开始 DFS 遍历;
\item 每次移动到新节点时,移除当前节点与目标节点之间的墙,并将目标节点加入连通图;
\item 递归完成后,整个网格形成一棵生成树,对应一个无环连通迷宫。
\end{itemize}

\subsection{迷宫生成算法-prim}
Prim 算法原本用于求解最小生成树。在迷宫生成中,我们将其改造为随机 Prim 算法:不考虑边权,而是随机选择待扩展的边。\par

算法维护一个“边界集合”,包含所有与已生成区域相邻但尚未加入的格子。
每一步从边界集中随机选取一个格子,将其与已生成区域中的某个邻居连接(打通墙壁),
并将该格子的新邻居加入边界集。重复此过程直至覆盖全部格子。\par

\begin{itemize}
\item 将网格视为完全图,每条潜在通道(相邻格子间)是一条边;
\item 任选一个起始格子加入生成树;
\item 将其所有相邻格子加入“边界集”;
\item 从边界集中随机选一格子,随机选择其一个已在生成树中的邻居,打通两者之间的墙,并将该格子纳入生成树,同时将其未访问邻居加入边界集;
\item 直至所有格子被纳入。
\end{itemize}

DFS生成的迷宫通常具有长而曲折的走廊和较少的分支,Prim 生成的迷宫通常具有更多短分支和较短的主路径。
\begin{figure}[H]
    \centering
    \includegraphics[scale = 0.2]{figure/dfs.png}
    \caption{DFS生成迷宫}
\end{figure}
\begin{figure}[H]
    \centering
    \includegraphics[scale = 0.2]{figure/prim.png}
    \caption{prim生成迷宫}
\end{figure}

\subsection{解迷宫算法-A*}
A*算法是一种广泛应用于路径规划和图遍历的启发式搜索算法。
它结合了 Dijkstra 算法的完备性与贪心最佳优先搜索的效率,
能够高效地找到从起点到终点的最短路径。\par

A* 算法通过评估函数 f(n) 来决定下一个要探索的节点:
$$
f(n)=g(n)+h(n)
$$
其中:
\begin{itemize}
\item g(n):从起点到当前节点 n 的实际代价。
\item h(n):从当前节点 n 到目标点的启发式估计代价。
\end{itemize}
在本项目中使用曼哈顿距离$abs(x_1 - x_2) + abs(y_1 - y_2)$进行启发式搜索。
我们在初始化时将起点(0,0)加入open列表中,然后进入循环,直到open列表为空或找到终点为止。
在每一步迭代中,我们从open列表中选择f值最小的节点作为当前节点。
然后检查当前节点是否为目标节点,如果是则结束循环;如果不是目标节点,
则将其标记为已处理并将其相邻的未处理节点加入open列表中。
同时,将已处理的节点加入closed列表中以避免重复处理。
\begin{figure}[H]
    \centering
    \includegraphics[scale = 0.2]{figure/astar.png}
    \caption{A*算法解迷宫}
\end{figure}

\subsection{自定义深度学习框架 (simple\_nn)}
为了替代PyTorch,我们实现了\texttt{simple\_nn.py}。
\subsubsection{卷积层实现}
卷积操作的核心在于高效地提取局部特征。为了避免低效的多重循环,
采用了\texttt{im2col}技术,将输入特征图展开为矩阵,
从而将卷积运算转化为矩阵乘法。
\begin{equation}
    Y = W_{col} \times X_{col} + b
\end{equation}
其中 $X_{col}$ 是展开后的输入矩阵,$W_{col}$ 是重排后的权重矩阵。

\subsubsection{优化器}
实现了Adam优化器,结合了动量法和RMSProp的优点,自适应调整学习率。
\begin{align}
    m_t &= \beta_1 m_{t-1} + (1-\beta_1) g_t \\
    v_t &= \beta_2 v_{t-1} + (1-\beta_2) g_t^2 \\
    \theta_{t+1} &= \theta_t - \frac{\eta}{\sqrt{\hat{v}_t} + \epsilon} \hat{m}_t
\end{align}

\subsection{AI训练-DQN}
DQN (Deep Q-Network) 将卷积神经网络与Q-Learning结合,解决了高维状态空间的强化学习问题。
\subsubsection{核心机制}
\begin{itemize}
    \item 经验回放:构建一个回放池 $D$,存储转移样本 $(s, a, r, s', done)$。训练时随机采样一个小批量,打破了数据间的相关性,提高了训练的稳定性。
    \item 目标网络:引入一个参数滞后的目标网络 $\hat{Q}$ 计算目标值,避免了“自举”导致的目标值震荡。目标网络每一定时间从策略网络中复制参数。
\end{itemize}
通过贝尔曼最优方程来计算期望Q值,即执行该动作后获得的即时奖励 r 加上对未来最大 Q 值的折扣期望
$$
Q^*(s, a) = \mathbf{E}*{s' \sim \mathcal{P}} \left[ r + \gamma \max*{a'} Q^*(s', a') \right]
$$
目标函数为最小化均方误差:
\begin{equation}
    L(\theta) = \mathbf{E}\left[ \left( r + \gamma \max_{a'} \hat{Q}(s', a'; \theta^-) - Q(s, a; \theta) \right)^2 \right]
\end{equation}

\begin{figure}[H]
    \centering
    \includegraphics[scale = 0.2]{figure/dqn.png}
    \caption{dqn算法流程结构}
\end{figure}

\subsubsection{Q值估计网络设计-输入层}
输入层设计为 $7 \times H \times W$ 的张量,包含7个通道:
\begin{itemize}
    \item 通道 0-3 (墙壁信息):分别编码每个格子上、下、左、右四个方向的墙壁存在情况
    \item 通道 4-5 (位置信息):分别使用One-hot编码智能体位置和目标位置
    \item 通道 6 (历史轨迹):记录智能体访问过的路径,赋予智能体记忆
\end{itemize}

\subsubsection{Q值估计网络设计-卷积层}
网络包含4层卷积层,通道数分别为 32, 64, 128, 128,主要为卷积层与relu激活层的循环。
网络中没有加入传统CNN网络中常见的池化层,主要是因为池化层的加入会模糊智能体与目标点的位置信息,
影响训练效果。\par
通过堆叠4层卷积层,深层神经元的感受野逐渐扩大,能够感知更大范围的迷宫结构,从而规划长距离路径。
通道数逐层增加(32 $\to$ 128),使得网络能够组合低级几何特征.
\begin{table}[htbp]
\centering
\caption{卷积神经网络各层参数配置}
\begin{tabular}{|l|c|c|c|c|c|}
\hline
\textbf{} & \textbf{输入通道数} & \textbf{输出通道数} & \textbf{卷积核大小} & \textbf{步长} & \textbf{填充} \\ \hline
第一层    & 7                   & 32                  & 3                    & 1           & 1           \\ \hline
第二层    & 32                  & 64                  & 3                    & 1           & 1           \\ \hline
第三层    & 64                  & 128                 & 3                    & 1           & 1           \\ \hline
第四层    & 128                 & 128                 & 3                    & 1           & 1           \\ \hline
\end{tabular}
\end{table}

\subsubsection{Q值估计网络设计-全连接层}
卷积层输出展平后,连接3层全连接层(1024 $\to$ 512 $\to$ 4),提供了
强大的非线性拟合能力,将提取的迷宫特征转化为上、下、左、右四个离散动作的Q值。

\subsection{超参数设计}
\begin{lstlisting}
# Hyperparameters
WIDTH, HEIGHT = 10, 10
EPISODES = 10000 
BATCH_SIZE = 32
GAMMA = 0.9 # 对未来奖励的关心程度,越大越关心
# 使用线性下降的EPSILON,EPSILON越大越倾向于随机探索
EPSILON_START = 1.0
EPSILON_END = 0.05
EPSILON_DECAY_EPISODES = 4000
LR = 0.0001 
TARGET_UPDATE = 200 #目标网络的更新频率
MEMORY_SIZE = 50000 #经验池的大小
\end{lstlisting}

\subsection{AI训练-遗传算法}
遗传算法是一种受达尔文生物进化论启发的元启发式优化算法。它模拟了自然界“物竞天择,适者生存”的过程,通过在解空间中维护一个种群(Population),并利用选择(Selection)、交叉(Crossover)和变异(Mutation)等遗传算子,迭代地演化出越来越优的解决方案。\par

在这个项目中,我们使用一个全连接神经网络来对智能体的下一步动作进行决策,通过遗传算法迭代更新神经网络的参数。
同时在训练过程中,我们借鉴了DQN的训练思路,每隔一段时间将训练目标的参数从智能体中复制下来,以求获得更加强大的智能体。

\begin{itemize}
    \item 选择算子:锦标赛选择
    \item 交叉算子:均匀交叉
    \item 变异算子:高斯扰动
    \item 进化策略:精英保留,将每一代中适应度最高的前10\%个体(精英)直接复制到下一代,不经过交叉和变异
\end{itemize}

\subsection{连续环境下的智能决策系统}
随着游戏环境从离散网格升级为连续物理环境,AI的决策机制也进行了相应的升级。
新的智能体不再局限于上下左右的离散移动,而是基于连续的坐标体系和角度控制,
实现了更加精准的战术动作。

\subsubsection{基于物理场的危险感知模型}
在连续环境中,子弹的轨迹不再是简单的网格跳跃,而是具有精确坐标
和速度矢量的连续运动。AI引入了危险评分机制,通过模拟场上所有子弹在未
来时间窗口 $T_w = 1.5s$ 内的运动轨迹,计算其与自身碰撞的风险。

对于每一个子弹 $b_i$,我们首先将其相对速度 $v_{rel}$ 和相对位置投影
到坦克的局部坐标系中。利用Liang-Barsky算法或分离轴定理,检测子弹轨
迹射线与坦克有向包围盒的相交情况。为了增加安全性,我们在OBB检测中
引入了安全边距 $margin = 5$ 像素。若发生碰撞,则计算预计碰
撞时间 $t_{impact}$。危险分数 $S_{danger}$ 定义为所有威胁子弹分数的最大值:
\begin{equation}
    S_{danger} = \max_{i \in \mathcal{B}_{threat}} \left( \frac{1}{t_{impact}^{(i)} + \epsilon} \right)
\end{equation}
其中 $\mathcal{B}_{threat}$ 为所有在 $T_w$ 内会发生碰撞的子弹
集合,$\epsilon=0.1$ 为防止除零的平滑项。该公式表明,碰撞时间越短,危险分
数越高,呈反比关系。

\subsubsection{分层行为树与动态规避}
为了在复杂的实时对战中做出最优反应,AI采用了基于优先级的行为树
决策逻辑。系统在每一帧计算当前状态的危险分数,并据此动态切换行为模式:

\begin{enumerate}
    \item \textbf{生存优先}:当检测到 $S_{danger} > 0$ 或预判动作会导
    致危险时,触发规避逻辑。系统采样动作
    空间 $\mathcal{A} = \{Forward, Backward, Stop\} \times \{Left, Right, None\}$ 中的9种候选
    动作,分别模拟执行后的未来状态并计算危险分数,选择 $S_{danger}$ 最小的动作执行:
    \begin{equation}
        a^* = \arg\min_{a \in \mathcal{A}} S_{danger}(s', a)
    \end{equation}
    
    \item \textbf{攻击决策}:若处于安全状态且满足射击条件,则进入
    攻击模式。系统计算目标方位角 $\theta_{target}$ 与当前朝
    向 $\theta_{current}$ 的偏差 $\Delta \theta$,采用比例控制策略调整朝向:
    \begin{equation}
        u_{turn} = \begin{cases} 
        Right, & \text{if } \Delta \theta > \delta \\
        Left, & \text{if } \Delta \theta < -\delta \\
        None, & \text{otherwise}
        \end{cases}
    \end{equation}
    其中 $\delta=0.1$ rad 为瞄准死区。当 $|\Delta \theta| < 0.15$ rad 时,AI将判定为锁定目标并自动开火。
    
    \item \textbf{路径规划}:作为默认行为,规划
    通向敌人的路径。在路径跟踪控制中,我们采用了分段控制策略:
    \begin{equation}
        u_{move} = \begin{cases} 
        Turn, & \text{if } |\Delta \theta| > 0.5 \text{ rad} \\
        Forward + Turn, & \text{if } |\Delta \theta| \le 0.5 \text{ rad}
        \end{cases}
    \end{equation}
    该策略保证了在大幅度转向时原地旋转以避免碰撞,而在小
    角度偏差时能够边走边转,实现平滑的切向移动。
\end{enumerate}

\subsubsection{视线检测与通视性分析}
为了判断能否攻击敌人或进行路径平滑,系统实现了基于几何的通视
性检测算法。该算法将连接起点与终点的线段视为射线,遍历场景中所有的静态墙
壁和动态墙壁。

对于每一个墙壁矩形,将其适度膨胀以预留安全边距,然后
利用Liang-Barsky线段裁剪算法检测射线是否与矩形相交。若无
任何相交,则判定两点间通视。该检测被广泛应用于攻击判定和路
径优化中。

\subsubsection{混合路径规划算法}
针对连续环境下的移动需求,设计了“Dijkstra全局规划 + 路径平滑”的混合路径规划算法:
\begin{itemize}
    \item \textbf{全局规划}:在底层的网格地图上运行Dijkstra算法。由于网格图边权均等,该算法等价于广度优先搜索,能够保证找到从起点格子到终点格子的最短离散路径 $P_{grid} = \{c_0, c_1, ..., c_n\}$。
    \item \textbf{路径平滑}:由于网格路径呈锯齿状,不符合坦克的运动学特性。系统采用“拉线法”对路径进行后处理:从路径起点 $p_{start}$ 开始,贪心地尝试连接后续节点 $p_k$。若线段 $\overline{p_{start}p_k}$ 不与任何墙壁相交,则剔除中间节点,将路径拉直。
    \item \textbf{路径跟踪控制}:智能体实时跟踪平滑后的路径点。控制律设计如下:若角度偏差 $|\Delta \theta| > 0.5$ rad,则原地旋转;若 $|\Delta \theta| \le 0.5$ rad,则在调整角度的同时全速前进,实现平滑的切向移动。
\end{itemize}

连续环境下整体场景如下图所示:
\begin{figure}[H]
    \centering
    \includegraphics[scale = 0.15]{figure/tree.png}
    \caption{连续环境下的坦克对战场景}
\end{figure}    

\section{主要仪器设备}
\begin{itemize}
    \item \textbf{开发语言}:Python 3.10
    \item \textbf{核心库}:NumPy (用于矩阵运算), Matplotlib (用于可视化)
    \item \textbf{硬件环境}:AMD CPU
    \item \textbf{操作系统}:Linux/Windows
\end{itemize}

\section{设计结果}
对于简单环境的DQN训练,reward曲线如下图所示
\begin{figure}[H]
    \centering
    \includegraphics[scale = 0.4]{figure/rewards.png}
    \caption{dqn奖励变换曲线}
\end{figure}
更多结果请参见代码压缩包与视频压缩包

\section{结论}
本报告详细阐述了从零构建深度强化学习框架及坦克对战AI的全过程。我们首先成
功实现了基于NumPy的底层神经网络框架,并通过DQN算法在离散迷宫中完成了初
步的寻路训练,验证了“去框架化”深度学习的可行性。

同时,遗传算法的实验结果表明,在特定任务下,基于进化的参数搜索策略也能取得与梯度下降相媲美的效果,为强化学习提供了有益的补充。

随后,针对实战中复杂的连续物理环境,我们突破了离散动作空间的限制,开发了基于
行为树、危险势场及混合路径规划的Smart AI。该系统成功解决了连续空间下的碰撞
检测、动态规避及平滑控制等难题。实验结果表明,Smart AI在动态规避子弹和战术进
攻方面表现优异,展现了启发式算法在实时对抗类游戏中的鲁棒性与高效性。

本项目不仅加深了对深度学习底层原理的理解,也探索
了将传统游戏AI技术与现代物理引擎结合的有效路径,为后续研究更复杂
的强化学习算法奠定了坚实基础。

\section{项目分工与产出}

\begin{table}[htbp]
\centering
\caption{人员分工}
\begin{tabular}{|c|c|c|c|}
\hline
\textbf{姓名} & \textbf{学号} & \textbf{分工}  & \textbf{贡献占比}\\ \hline
黄鹤翔    & 3230106231                   & 简单环境迷宫生成、决策ai、DQN训练    &10            \\ \hline
高恒斌    & 3230105132                  & 连续环境决策ai     &10           \\ \hline
都奕宁    & 3230100948                  & 简单环境决策、连续环境决策        &10      \\ \hline
\end{tabular}
\end{table}
本项目的完整实现代码、训练脚本已开源在GitHub仓库:
\url{https://github.com/GgbNnD/tankalg}


\section{参考文献}
\begin{enumerate}
    \item Mnih, V., Kavukcuoglu, K., Silver, D. et al. Human-level control through deep reinforcement learning. Nature 518, 529–533 (2015).
    \item Hart, P. E., Nilsson, N. J., \& Raphael, B. (1968). A Formal Basis for the Heuristic Determination of Minimum Cost Paths. IEEE Transactions on Systems Science and Cybernetics, 4(2), 100-107.
    \item Liang, Y. D., \& Barsky, B. A. (1984). A new concept and method for Line Clipping. ACM Transactions on Graphics (TOG), 3(1), 1-22.
    \item Holland, J. H. (1992). Adaptation in natural and artificial systems. MIT press.
    \item 河南葫季果, L\_Shy\_P. (2024). Bilibili Video: https://www.bilibili.com/video/BV17EswzLEKw/
\end{enumerate}


\section{引用}
连续环境下游戏引擎参考自GitHub开源项目:https://github.com/mglyn/TANKTROUBLE-pythonedition.git,
其依赖于numpy/pygame库

\end{document}
